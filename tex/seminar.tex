\documentclass[12pt]{beamer}

\usepackage[english]{babel}
\usepackage[utf8]{inputenc}
\usepackage{tikz}

\usetheme{Copenhagen}
\setbeamertemplate{navigation symbols}{}

\title{The FLP Theorem}
\author[Jacopo Notarstefano]{
  Jacopo Notarstefano\\
  \texttt{jacopo.notarstefano [at] gmail.com}
}
\date{}

\begin{document}
  \begin{frame}[plain]
    \titlepage
  \end{frame}

  \begin{frame}{The Distributed Consensus Problem}
    \begin{definition}{}
    \end{definition}
  \end{frame}

  \begin{frame}{Consensus Protocol}
  \end{frame}

  \begin{frame}{Message System}
  \end{frame}

  \begin{frame}{Partial correctness}
    \begin{definition}[Partial correctness]
      A consensus protocol is \textbf{partially correct} if:
      \begin{enumerate}
        \item No accessible configuration has more than one decision value.
        \item For each \(v\in \{0,1\}\), some accessible configuration has decision value \(v\).
      \end{enumerate}
    \end{definition}
  \end{frame}

  \begin{frame}{Total correctness in spite of one fault}
    A process \(p\) is \textbf{nonfaulty} in run if it takes infinitely many steps, otherwise it is \textbf{faulty}.

    \vspace{0.25cm}

    A run is \textbf{admissible} if at most one process is faulty and all messages sent to nonfaulty processes are eventually received.

    \vspace{0.25cm}

    A run is \textbf{deciding} if some process reaches a decision state.

    \vspace{0.25cm}

    \begin{definition}[Total correctness in spite of one fault]
      A consensus protocol \(P\) is \textbf{totaly correct in spite of one fault} if it is partially correct and every admissibile run is deciding.
    \end{definition}
  \end{frame}

  \begin{frame}{Main result}
    \begin{theorem}[Fischer, Lynch, Paterson 1985]
      No consensus protocol is totally correct in spite of one fault.
    \end{theorem}

    \begin{proof}[Proof (sketch)]
      \begin{enumerate}
        \item There is some initial bivalent configuration.
        \item We construct an admissible run that avoids ever taking a step that would commit the system to a decision.
      \end{enumerate}
    \end{proof}
  \end{frame}

  \begin{frame}{Lemma 1}
    \begin{lemma}
    \end{lemma}
    \begin{proof}
    \end{proof}
  \end{frame}

  \begin{frame}{Lemma 2}
    \begin{lemma}
    \end{lemma}
    \begin{proof}
    \end{proof}
  \end{frame}

  \begin{frame}{Lemma 3}
    \begin{lemma}
    \end{lemma}
    \begin{proof}
    \end{proof}
  \end{frame}

  \begin{frame}{Proof of main result}
    \begin{proof}
    \end{proof}
  \end{frame}
\end{document}
